\documentclass{article}

% Language setting
% Replace `english' with e.g. `spanish' to change the document language
\usepackage[polish]{babel}

% Set page size and margins
% Replace `letterpaper' with `a4paper' for UK/EU standard size
\usepackage[letterpaper,top=2cm,bottom=2cm,left=3cm,right=3cm,marginparwidth=1.75cm]{geometry}

% Useful packages
\usepackage{setspace}
\usepackage{amssymb}
\usepackage{amsmath}
\usepackage{amsfonts}
\usepackage{graphicx}
\usepackage{indentfirst}
\usepackage{algpseudocode}
\usepackage[T1]{fontenc}
\usepackage[mathscr]{euscript}
\usepackage[colorlinks=true, allcolors=blue]{hyperr ef}

\author{Hubert Michalski}

\begin{document}
\Large
\maketitle

\section{Podpunkt a)}

Najpierw wykażemy, że dopełnienie języka $L$ nie jest obliczalne, a z tego w szczególności wynika, że język $L$ też nie jest obliczalny. Załóżmy nie wprost, że język $L^C$ jest obliczalny. Wykażemy, za pomocą odpowiedniej redukcji, że wówczas język $HALT$ byłby obliczalny, co jest nieprawdą.

Weźmy zatem dowolną instancję $(u_\mathcal{M}, w)$ problemu $HALT$ i zredukujmy ją do instancji problemu $coSIMILAR$, czyli dopełnienia problemu $SIMILAR$. Oznaczmy przez $\#$ symbol, który nie występuje w alfabecie taśmowym maszyny $\mathcal{M}$. Dodatkowo niech $\mathcal{M}_{w \rightarrow \#}$ oznacza maszynę która działa identycznie jak $\mathcal{M}$ poza przypadkiem gdzie na wejście otrzymuje $w$, w takim wypadku nowa maszyna wypisywałaby na wyjście symbol $\#$ i terminowała. Można pokazać, że taką maszynę da się w prosty sposób skonstruować dodając jednego $"if'a"$ przed wywołaniem $\mathcal{M}$. Rozważmy więc następującą funkcję:
$$
(u_{\mathcal{M}}, w) \mapsto (u_{\mathcal{M}}, u_{\mathcal{M}_{w \rightarrow\#}})
$$

Widać z definicji, że taka funkcja jest obliczalna. Zbadajmy zatem jak zachowują się obie maszyny na słowie $w$ (na reszcie słów zachowują się identycznie). Zauważmy, że jeśli $\mathcal{M}$ terminuje na słowie $w$ to maszyny $\mathcal{M}$ i $\mathcal{M}_{w \rightarrow\#}$ nie uznamy za $podobne$, ponieważ pierwotna maszyna nie może wypisać znaku $\#$ na wyjście. W przeciwnym przypadku jeśli $\mathcal{M}$ nie terminuje na $w$ to widzimy, że predykat o terminowaniu obu maszyn nie jest spełniony, czyli maszyny uznamy za $podobne$. Łatwo więc zauważyć, że zachodzi równoważność:
$$
(u_{\mathcal{M}}, w) \in HALT \iff (u_{\mathcal{M}}, u_{\mathcal{M}_{w \rightarrow\#}}) \in coSIMILAR
$$

Zatem gdyby zachodziło $coSIMILAR=L(\mathcal{K})$ dla pewnej maszyny $\mathcal{K}$, to skonstruowalibyśmy maszynę $\mathcal{N}$ dla języka $HALT$, która dla słów postaci $(u_{\mathcal{M}}, w)$ oblicza słowo $(u_{\mathcal{M}}, u_{\mathcal{M}_{w \rightarrow\#}})$ i uruchamia na nim maszynę $\mathcal{K}$. Maszyna $\mathcal{N}$ akceptowałaby język $HALT$, co jest niemożliwe.$ \hfill \blacksquare$
\newpage
\section{Podpunkty b) i c)}

Udowodnimy, że dopełnienie języka $L$ jest częściowo obliczalne, czyli skonstruujemy maszynę, która dla wszystkich słów postaci $(u_{\mathcal{M}}, u_{\mathcal{N}})$ z języka $L^C$ będzie terminowała i mówiła, że słowo należy do języka a dla reszty będzie się wykonywać w nieskończoność. Rozważmy maszynę $\mathcal{K}$ (nazwijmy ją nadzorującą) która będzie uruchamiać nowe instancje maszyn $\mathcal{M}$ i $\mathcal{N}$ na słowach $w \in \{0,1\}^*$. Żeby w konsekwentny sposób wybierać kolejne słowa do sprawdzenia można skonstruować pod-procedurę która liczy poniższą funkcję $f: \mathbb{N} \rightarrow \{0,1\}^*$:
$$
f(n) = \begin{cases}
\varepsilon & n = 0 \\
bin(n-1) & n \in \mathbb{N}-\{0\} \\
\end{cases}
$$

Schemat działania maszyny nadzorującej będzie następujący: dla indeksu pętli $i=0,1,2\ldots $ będzie ona uruchamiać nowe instancje maszyn $\mathcal{M}$ i $\mathcal{N}$ na $f(i)$ oraz wykonywać jeden ruch na wszystkich dotychczasowo uruchomionych maszynach, które jeszcze nie terminowały. Jeśli jakieś dwie maszyny które zostały uruchomione na tym samym słowie zakończą działanie i wypiszą na wyjście różne słowa, to znaczy, że maszyny $\mathcal{M}$ i $\mathcal{N}$ nie są $podobne$, ponieważ znaleźliśmy świadka który na to wskazuje. Zauważmy, że postępując w ten sposób uruchomimy przeliczalnie wiele maszyn oraz dla danego słowa $w$ zostaną kiedyś uruchomione na nim obie maszyny z wejścia, o ile wcześniej $\mathcal{K}$ się nie zatrzyma. Zatem pod warunkiem, że słowo $(u_{\mathcal{M}}, u_{\mathcal{N}})$ należy do języka $L^C$ to uruchamiając maszynę nadzorującą $\mathcal{K}$ kiedyś się o tym dowiemy, czyli język ten jest częściowo obliczalny.
\hfill \break

Otrzymujemy więc, że język $L$ nie jest częściowo obliczalny, ponieważ gdyby $L$ był częściowo obliczalny to by oznaczało, że jest on także obliczalny co prowadzi do sprzeczności.$ \hfill \blacksquare$




\end{document}
