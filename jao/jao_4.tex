\documentclass{article}

% Language setting
% Replace `english' with e.g. `spanish' to change the document language
\usepackage[polish]{babel}

% Set page size and margins
% Replace `letterpaper' with `a4paper' for UK/EU standard size
\usepackage[letterpaper,top=2cm,bottom=2cm,left=3cm,right=3cm,marginparwidth=1.75cm]{geometry}

% Useful packages
\usepackage{setspace}
\usepackage{amssymb}
\usepackage{amsmath}
\usepackage{amsfonts}
\usepackage{graphicx}
\usepackage{indentfirst}
\usepackage{algpseudocode}
\usepackage[T1]{fontenc}
\usepackage[mathscr]{euscript}
\usepackage[colorlinks=true, allcolors=blue]{hyperr ef}

\author{Hubert Michalski}

\begin{document}
\Large
\maketitle

\section{Zadanie 4}

Wpierw zauważmy, że problem jest klasie NP. Istotnie: żeby go rozwiązać, niedeterministycznie zgadujemy wartościowanie $v$ a następnie bierzemy minimum z liczby literałów które są prawdziwe w każdej klauzuli przy danym wartościowaniu $v$, na koniec sprawdzamy czy liczba ta jest w zakresie od $3$ do $5$. Jest to algorytm niedeterministyczny działający w czasie wielomianowym.

Teraz pokażemy poprzez redukcję z problemu 3-CNF-SAT, który jest NP-zupełny, że rozważany problem jest NP-trudny. Wskażmy zatem wielomianowy algorytm, który przyjmuje formułę boolowską $\psi$ z 3-CNF i przekształca ją na taką formułę $\phi$ z CNF, że $3\leq score(\phi,v) \leq 5$ wtedy i tylko wtedy, gdy istnieje wartościowanie $v$ które spełnia formułę $\psi$. Rozważmy następującą transformację - dla danej formuły $\psi$ zastąpimy każdą jej klauzulę $C$ przez taką klauzulę $C'$, że będzie ona identyczna jak $C$ lecz dodatkowo ostatni literał będzie dodany na koniec czterokrotnie - dwa razy tak samo i dwa razy zanegowany.
Przykładowo:
$$
(p\lor q \lor r) \land (x\lor y) \mapsto
(p\lor q \lor r \lor r \lor r \lor \neg r \lor \neg r) \land (x\lor y \lor y \lor y \lor \neg y \lor \neg y)
$$
Oczywiście można pokazać, że taki algorytm da się zrealizować w czasie wielomianowym. Sprawdźmy więc, czy zachodzi wcześniej wspominana równoważność.

Łatwo zaobserwować, że dla formuły $\psi$ z 3-CNF i wartościowaniu $v$ $score(\psi, v)$ leży między zero a trzy, z kolei rozszerzenie każdej klauzuli $C$ o nowe literały dodaje dokładnie dwa $true$, ponieważ będą dodatkowo spełnione albo dwa sklonowane literały albo ich negacje, zatem dla nowej formuły z CNF zachodzi $2\leq score(\phi, v) \leq 5$. Przypuśćmy, że formuła $\psi$ jest spełnialna przez pewne wartościowanie $v$. To oznacza, że przy wartościowaniu $v$ w każdej klauzuli $C$ co najmniej jeden literał jest prawdziwy, tak więc w każdej klauzuli $C'$ formuły $\phi$ są co najmniej trzy prawdziwe literały, czyli istnieje takie wartościowanie, że $3\leq score(\phi, v) \leq 5$.
\newpage

Z drugiej strony załóżmy, że istnieje wartościowanie $v$ takie, że\break $3 \leq score(\phi,v) \leq 5$. W takim przypadku wiemy, że w każdej klauzuli $C'$ formuły $\phi$ co najmniej trzy literały są prawdziwe przy wartościowaniu $v$, przy czym dokładnie dwa z nich znajdują się w ostatnich czterech literałach. Otrzymujemy w ten sposób, że co najmniej jeden literał w każdej klauzuli $C$ formuły $\psi$ jest prawdziwy, zatem ta formuła jest spełnialna przez wartościowanie $v$.

Ostatecznie stworzyliśmy wielomianowy algorytm który zamienia wejścia problemu 3-CNF-SAT na równoważne wejścia naszego problemu, czyli nasz problem też jest NP-trudny. Udowodniliśmy także na początku, że należy on do NP, zatem problem ten jest NP-zupełny.

\end{document}
