\documentclass[final,12pt]{article}
\usepackage{a4wide}
\usepackage{amsmath}
\usepackage{amssymb}
\usepackage{latexsym}
\usepackage[polish,english]{babel}
\selectlanguage{english}
\usepackage[utf8]{inputenc}
\usepackage[T1]{fontenc}
\usepackage[pdftex]{graphicx}
\usepackage[pdftex]{color}
\usepackage[final]{listings}
\usepackage[a4paper,margin=0.8in,footskip=0.25in]{geometry}


%%%%%%%%%%%%% listings
\lstdefinelanguage{whileprograms}{morekeywords={while,do,if,then,else,=),(,decr,in,wrt},%
   sensitive,%
   morecomment=[l]//,%
   morecomment=[s]{\{}{\}},%
   morecomment=[s]{[}{]},%
    basicstyle=\small\tt,
    keywordstyle=\normalfont\bfseries\color{black},
    commentstyle=\color{blue},
    mathescape=true,
}

\lstset{language=whileprograms,flexiblecolumns=true,mathescape=true,frame=none}
\lstset{commentstyle=\it,basicstyle=\tt}
\lstset{literate={<=}{{$\leq \ $}}1
                 {>=}{{$\geq$}}1
                 {^2}{{$^2$}}1
                 {^k}{{$^k$}}1
                 {EEK}{{$\exists_{k \in \mathbb{N}.} $}}1
                 {&}{{$\land \ $}}1}

\usepackage[final]{listings}

\pagestyle{empty}

\title{\vspace{-1cm}
Semantyka i Weryfikacja programów \\
\large Praca domowa 3.
}
\author{Hubert Michalski hm438596}

\begin{document}
\lstset{language=whileprograms} \selectlanguage{polish}

\maketitle
\section{Zadanie}
Na kolejnej stronie podany jest program zapisany w języku $TINY$ rozszerzonym o operację \textbf{div2} dzielenia całkowitego przez 2 oraz, na potrzeby formułowania asercji, o operację podnoszenia liczb całkowitych do całkowitej nieujemnej potęgi. Jak widać z podanej specyfikacji, jest to kolejna wersja liczenia pierwiastka całkowitego liczby całkowitej dodatniej. Udowodnij częściową poprawność tego programu względem podanej specyfikacji, podając niezmienniki obu pętli oraz wstawiając odpowiednie formuły w nawiasy \{. . .\} tak, aby podane niezmienniki i asercje zapisały przeprowadzony dowód częściowej poprawności programu w logice Hoare'a. Jeśli w dwóch sąsiednich wierszach występują nawiasy \{. . .\} to pomiędzy wstawionymi tam asercjami powinna zachodzić implikacja. Można też dodać dodatkowe nawiasy i wpisać w nie odpowiednie asercje. Poza niezmiennikami $\gamma_1$ i $\gamma_2$ , wymagane jest przynajmniej podanie formuł $\alpha_1, \alpha_2, \alpha_3, \alpha_4, \alpha_5$ (ale ewentualne błędy w innych
formułach też będą wpływały na ostateczną ocenę rozwiązania).

\newpage
\section{Rozwiązanie}

\begin{lstlisting}
{ n > 0 }
i := 1;
{ n > 0 & i = 1 }
kw := 4;
{ n > 0 & i = 1 & kw = 4 }
while {$\gamma_1$: kw = 4i^2 & i^2 <= n & EEK i = 2^k }
    kw <= n
  do
  ({ kw = 4i^2 & 4i^2 <= n & EEK 2i = 2^k }
   i := 2*i;
   {$\alpha_1$: kw = i^2 & i^2 <= n & EEK i = 2^k }
   kw := 4*kw
   { kw = 4i^2 & i^2 <= n & EEK i = 2^k }
  )
{$\alpha_2$: kw = (2i)^2 & i^2 <= n < kw & EEK i = 2^k }
r := i;
{ kw = (r + i)^2 & r = i & r^2 <= n < kw & EEK i = 2^k }
dri := kw div2;
{ kw = (r + i)^2 & r = i & dri = 2ri & r^2 <= n < kw & EEK i = 2^k }
ik := dri div2;
{ kw = (r + i)^2 & r = i & dri = 2ri & ik = i^2 & r^2 <= n < kw & EEK i = 2^k }
while {$\gamma_2$: kw = (r + i)^2 & dri = 2ri & ik = i^2 & r^2 <= n < kw & EEK i = 2^k }
     i > 1
  do
  ({ ik/4 = i^2/4 & dri/2 = 2ri/2 & kw = (r + i)^2 & r^2 <= n < kw & EEK i = 2^k & i > 1 }
   i := i div2;
   {$\alpha_3$: ik/4 = i^2 & dri/2 = 2ri & kw = (r + 2i)^2 & r^2 <= n < kw & EEK i = 2^k }
   ik := (ik div2) div2;
   { ik = i^2 & dri/2 = 2ri & kw = (r + 2i)^2 & r^2 <= n < kw & EEK i = 2^k }
   dri := dri div2;
   {$\alpha_4$: ik = i^2 & dri = 2ri & kw = (r + 2i)^2 & r^2 <= n < kw & EEK i = 2^k }
   if (kw - dri - 3*ik) <=  n
   then
      { ik = i^2 & dri + 2ik = 2(r+i)i & kw = (r + 2i)^2 & (r+i)^2 <= n < kw & EEK i = 2^k }
      r := r + i;
      {$\alpha_5$: ik = i^2 & dri + 2ik = 2ri & kw = (r + i)^2 & r^2 <= n < kw & EEK i = 2^k }
      dri := dri + 2*ik
      { ik = i^2 & dri = 2ri & kw = (r + i)^2 & r^2 <= n < kw & EEK i = 2^k }
   else
      { ik = i^2 & dri = 2ri & kw = (r + i)^2 + dri + 3ik & r^2 <= n < (r + i)^2 & EEK i = 2^k }
      kw := kw - dri - 3*ik;
      { ik = i^2 & dri = 2ri & kw = (r + i)^2 & r^2 <= n < kw & EEK i = 2^k }
  )
{ r^2 <= n < (r + i)^2 & EEK i = 2^k & i <= 1 }
{ r^2 <= n < (r + 1)^2 }
\end{lstlisting}





\end{document}
