\documentclass[a4paper]{article}

\usepackage{amssymb,mathrsfs,amsmath,amscd,amsthm}
\usepackage[mathcal]{euscript}
\usepackage{stmaryrd}
\usepackage[T1]{fontenc}
\usepackage[utf8]{inputenc}
\usepackage[polish]{babel}
\usepackage{graphics}
\usepackage{blindtext}
\usepackage{enumitem}
\usepackage{amsmath}
\usepackage{algorithm}
\usepackage[noend]{algpseudocode}
\usepackage{mathtools}

\makeatletter
\makeatother

\RequirePackage{a4wide}

%%%%%%% makra do notacji

\renewcommand{\le}{\leqslant} %mniejsze bądź równe
\renewcommand{\ge}{\geqslant} %większe bądź równe
\renewcommand\qedsymbol{\scalebox{0.75}{$\blacksquare$}} %koniec dowodu
\newcommand\exendsymbol{\scalebox{1}{$\lrcorner$}} %inny koniec dowodu
\renewcommand{\phi}{\varphi} %litera φ
\newcommand{\eps}{\varepsilon} %litera ε

\newcommand{\N}{\mathbb N} %liczby naturalne
\newcommand{\R}{\mathbb R} %liczby rzeczywiste
\newcommand{\I}{\mathbb I} %liczby rzeczywiste
\newcommand{\set}[1]{\{#1\}} %\set{1,2,3} to zbiór {1,2,3}
\newcommand{\setof}[2]{\{#1\mid #2\}} %\setof{(x,y)}{x,y\in\N,x+y=5} to {(x,y)|x,y∈N, x+y=5}
\newcommand{\from}{\colon} %f\from X\to Y to funkcja f:X→Y

\renewcommand{\subset}{\subseteq} %symbol ⊆
\newcommand{\Longupdownarrow}{\Big\Updownarrow}

\newtheorem{twierdzenie}{Twierdzenie}
\newtheorem{fakt}{Fakt}
\newtheorem{wniosek}{Wniosek}
\newtheorem{lemat}{Lemat}
\newtheorem{zadanie}{Zadanie}
\newtheorem{zadanie*}{Zadanie$^*$}

\title{\vspace{-1cm}
    Metody Numeryczne \\
    \large Praca domowa 1.
}
\author{
    Hubert Michalski hm438596 \\
}

\begin{document}

\maketitle

\section*{Zadanie 1.1}

Dla silnie diagonalnie dominującej macierzy $A \in R^{N \times N}$ w postaci $Hessenberga$, tzn. takiej, że $a_{ij}=0$ dla $i > j + 1$:

$$\begin{bmatrix}
* & * & \dots & * \\
* & * & \dots & * \\
& \ddots & \ddots & \vdots \\
&  & *     & * \\
\end{bmatrix}$$

sformułuj:
\begin{enumerate}[label=(\alph*)]
	\item
	      algorytm wyznaczający kosztem $O(N^2)$ jej rozkład $LU$;

	\item
	      algorytm wyznaczający dla zadanego $b \in R^N$ rozwiązanie $x$ układu równań $Ax=b$ kosztem $O(N^2)$.

\end{enumerate}


\section*{\large Rozwiązanie $a)$}

Do rozkładu macierzy $A$ można użyć zmodyfikowanego algorytmu $GEPP$. Warto jednak na początku zaznaczyć, że rozważana macierz $A$ jest silnie diagonalnie dominująca tzn. $$\forall_{k} \\|a_{k,k}|>\sum_{i \neq k}|a_{k,i}|$$ więc nie ma konieczności wykorzystywania zamiany wierszy. Gdyby jednak zadana macierz nie miała tej własności, to można by dodatkowo zaobserwować, że dla $k$-tego elementu diagonali mamy jedynie dwie możliwe wartości tzn. $a_{k,k}$ lub $a_{k+1,k}$ ponieważ wszystkie elementy $a_{i,k}$, dla $i > k + 1$ są zerami z definicji. Zatem gdyby $|a_{k+1,k}| > |a_{k,k}|$ to byśmy zmieniali wiersze $k$ i $k+1$. Teraz jednak ten krok może zostać pominięty.

W przypadku standardowego algorytmu $GEPP$ kolejnym krokiem byłoby wyznaczenie $k$-tej kolumny macierzy $L$, w tym celu podzielilibyśmy jej elementy pod diagonalą przez $a_{k,k}$. Jednak dla naszego specjalnego przypadku wystarczy jedynie zmodyfikować wartość $a_{k+1,k}$, ponieważ (ponownie z definicji) reszta elementów tej kolumny to zera.

Następnie należy zaktualizować pozostałą część macierzy, biorąc pod uwagę poprzednie modyfikacje. Z poprzedniego kroku wiadomo, że zmieniła się tylko wartość $a_{k+1,k}$, co oznacza, że wystarczy zaktualizować wyłącznie $k+1$-szy wiersz macierzy $A$. Powyższe kroki powtarzamy dla $k = 1 : N - 1$.

\begin{algorithm}
	\begin{algorithmic}[1]
		\caption{LU decomposition for Hessenberg matrix}\label{alg:cap1}
		\For{k = 1 : $N - 1$}
		\State $a_{k+1,k} \gets a_{k+1,k} / a_{k,k}$

		\For{i = k + 1 : $N$}
		\State $a_{k+1,i} \gets a_{k+1,i} - a_{k+1,k}a_{k,i}$
		\EndFor
		\EndFor
	\end{algorithmic}\label{alg:algorithm1}
\end{algorithm}

\newpage
Szacowany koszt:
\begin{itemize}
	\item
	      liczba iteracji: $\mathcal{O}(N)$

	\item
	      aktualizacja wiersza w $k$-tej iteracji: $\mathcal{O}(N)$

\end{itemize}

Ostatecznie otrzymujemy: $\mathcal{O}(N^2)$

\section*{\large Rozwiązanie $b)$}

Aby wyznaczyć rozwiązanie zadanego układu równań skorzystajmy z wyżej wymienionego algorytmu do przedstawienia macierzy $A$ jako iloczynu macierzy $L$ oraz $U$. Dodatkowo wiemy, że równania z macierzami trójkątnymi można rozwiązać w czasie $\mathcal{O}(N^2)$. Wystarczy zatem dwukrotnie zastosować ten fakt do rozłożonej poprzednio macierzy i otrzymujemy rozwiązanie zadania:

\begin{algorithm}
	\begin{algorithmic}[1]
		\caption{Solve equation $Ax=b$ for Hessenberg matrix}\label{alg:cap2}
		\State $L,U \gets decompose(A)$ // $\mathcal{O}(N^2)$ z poprzedniego zadania, dla uproszczenia zapisu jako dwie macierze lecz nie zmienia to złożoności rozwiązania
		\State $y \gets L^{-1}b$ // Rozwiąż $Ly=b$, czas $\mathcal{O}(N^2)$
		\State $x \gets U^{-1}y$ // Rozwiąż $Ux=y$, czas $\mathcal{O}(N^2)$
	\end{algorithmic}\label{alg:algorithm2}
\end{algorithm}
Każdy krok ma złożoność $\mathcal{O}(N^2)$ zatem złożoność całego algorytmu to $\mathcal{O}(N^2)$.

\end{document}
